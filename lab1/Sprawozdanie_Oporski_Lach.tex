\documentclass[a4paper,12pt]{article}

\usepackage[utf8]{inputenc}
\usepackage[T1]{fontenc}
\usepackage[polish]{babel}
\usepackage{amsmath, amssymb}
\usepackage{graphicx}
\usepackage{geometry}
\usepackage{hyperref}

\geometry{margin=2.5cm}

\title{Sprawozdanie z Laboratorium \\[1ex] \large Tytuł Ćwiczenia}
\author{Imię Nazwisko \\ Numer indeksu: XXXXX \\ Prowadzący: dr inż. XYZ}
\date{\today}

\begin{document}

\maketitle

\section{Wstęp}
Krótki opis celu ćwiczenia i stosowanych metod.

\section{Stanowisko i aparatura}
Opis stanowiska laboratoryjnego i wykorzystanych przyrządów.

\section{Przebieg ćwiczenia}
Szczegółowy opis wykonanych czynności w ramach eksperymentu.

\section{Wyniki pomiarów}

%\begin{figure}[h!]
    %\centering
    %\includegraphics[width=0.7\textwidth]{wykres.pdf}
    %\caption{Przykładowy wykres uzyskany podczas ćwiczenia.}
    %\label{fig:wykres}
%\end{figure}

\section{Analiza i dyskusja}
Analiza otrzymanych wyników, porównanie z teorią, omówienie potencjalnych źródeł błędów.

\section{Wnioski}
Główne wnioski wynikające z przeprowadzonych badań.

\end{document}
